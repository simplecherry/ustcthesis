\chapter{绪论}

\section{课题背景与研究意义}

火电是我国能源结构的重要组成部分。2016年,我国火力发电量占全国发电量的72.2$\,$\si{\percent},且增速由2015年的下降2.6$\,$\si{\percent} 转为增长3.6$\,$\si{\percent}。水电虽然占发电量的19.4$\,$\si{\percent},但这种发电方式受季节和年份影响较大,且会对当地的自然环境产生一定影响。其余的核电、风电和太阳能发电一共占比8.4$\,$\si{\percent},虽然增速较快,但很难在短时间内影响火电在能源结构中的地位\cite{2016年能源生产情况}。

由于我国油气资源储量较少,在火电行业中,煤电占据着绝对的比重。但随着近年来大气污染越发严重,环保压力日益增加,煤电行业的污染问题也受到了国家和社会空前的关注。为此,国家部委相继出台《火电厂大气污染物排放标准》、《煤电节能减排升级与改造行动计划(2014-2020 年)》、《全面实施燃煤电厂超低排放和节能改造工作方案》等相关文件,要求燃煤电厂污染物排放达到燃气轮机组排放标准,且供电煤耗降低到310克标煤/千瓦时,淘汰落后的煤电机组,对改造后仍不符合能效环保要求的机组要重点淘汰\cite{中国环境科学研究院2012火电厂大气污染物排放标准,煤电节能减排升级与改造行动计划,关于印发《全面实施燃煤电厂超低排放和节能改造工作方案》的通知}。

在此背景下,研究、发展煤的清洁、高效燃烧技术就成为燃煤电厂减少污染物排放、降低发电煤耗的重中之重。目前,IGCC(Integrated Gasification Combined Cycle,整体式煤气化联合循环发电技术)、USCPC-FGD(Ultra Super Critical Pulverized Coal-Flue Gas Desulfurization,超超临界煤粉炉附加烟气脱硫技术)和CFBC(Circulating Fluidized Bed Combustion,循环流化床燃烧)是我国重点发展的洁净燃烧技术\cite{隋建才2008,陈昌和2010}。

IGCC技术需要将煤在高压氧气中部分氧化,生成气体燃料,之后气体燃料通过燃气轮机生产电力。这种燃烧方式污染物排放最少,其原始污染物排放量已经达到最新国标要求。但目前IGCC技术尚缺乏成熟运行的经验,装置复杂,气体燃料的生成需要大量的氧,发电成本较高。USCPC-FGD技术主要用于直流煤粉炉,通过提高蒸汽的温度、压力参数来提高锅炉的热效率,达到降低煤耗、减少~\ce{CO2}排放量的目的。但采用这种技术的锅炉仍然具有煤粉炉的缺点:热流分布不均匀,\ce{SO2}、\ce{NOx}(Nitrogen  Oxide,氮氧化物)排放量较高,开放式火焰易导致结渣等等。USCPC产生的污染物处理主要依靠附加的脱硫、脱硝、除尘设备,且高蒸汽参数对锅炉的材料有更高的要求。相比于USCPC-FGD,CFBC技术低温燃烧的特性有利于脱硫,不利于氮氧化物生成,可以显著降低锅炉原始污染物排放量。除了氮氧化物生成量低、炉内直接脱硫等优点外,循环流化床锅炉燃烧技术还具有燃料适应性广、燃烧效率和燃烧强度高、负荷调节范围大等优势。循环流化床锅炉还被用来燃烧城市垃圾和农作物等生物质燃料。此外,由于我国煤炭资源中高硫、高灰煤比重较大,发电用煤多为低挥发分煤,特别适合采用循环流化床技术进行燃烧。在《国家能源局关于促进低热值煤发电产业健康发展的通知》中,建议以煤矸石为主要燃料的燃煤机组优先选用循环流化床锅炉\cite{国家能源局关于促进低热值煤发电产业健康发展的通知}。

自1985年吉林锅炉厂研制出了国内第一台循环流化床锅炉后,我国以技贸结合的方式相继引进、吸收Alstom、Foster Wheeler等多家国外循环流化床锅炉设计厂家的技术\cite{周一工2009}。哈尔滨锅炉厂、东方锅炉厂、上海锅炉厂、无锡锅炉厂等锅炉生产厂家联合中科院工程热物理所、国电热工研究院、清华大学等科研机构开展了国内自主的知识产权的循环流化床锅炉的设计、制造相关研究。2013年4月,四川白马电站600$\,$\si{\mega\watt}~循环流化床示范电站正式投入商业运行,这是当时世界容量最大的超临界循环流化床示范电站工程,机组采用由东方锅炉厂自主开发设计制造的锅炉。截至2111年,我国循环流化床锅炉机组容量已经居世界首位。\cite{岳光溪2016}。

循环流化床锅炉燃烧系统的稳定运行是维持其低排放、高效率的关键,但其燃烧机理复杂,目前大多数锅炉的自动投运率较低。因此,研究、设计并实现CFB锅炉的燃烧系统的先进控制,不仅可以提高现场自动化水平、减轻操作人员工作强度,还可以在深入理解CFB锅炉的燃烧特性的基础上进一步提高锅炉的燃烧效率、降低污染物排放。

\ce{NOx}~是循环流化床锅炉排放的主要污染物之一,严重威胁自然生态环境和人民群众的身体健康。建立~\ce{NOx}软测量模型不仅可以用于锅炉污染物排放量的在线监测,降低监测成本,也可以用于低排放量约束下的锅炉燃烧优化,提高燃煤电厂的经济效益。


\section{国内外研究现状}

\subsection{循环流化床锅炉燃烧系统控制研究现状}

有关循环流化床锅炉燃烧系统控制的研究主要集中在三个方面:燃烧系统建模、控制策略设计、多目标优化。完善的燃烧系统模型有助于工程技术人员了解循环流化床锅炉的运行燃烧机理,从而设计更合适的控制策略,提高锅炉的控制水平。在实现锅炉自动控制的基础上,多目标优化技术可以改善系统的运行状态,使之达到更好的经济、环保指标。

燃烧系统建模方法包括机理建模和数据驱动建模两种。循环流化床锅炉的燃烧机理非常复杂,机理建模法通过对锅炉内的煤粒分布做一定假设,结合燃烧、传热、质量守恒和能量守恒等相关原理建立锅炉的数学模型。机理建模得到的模型虽然有充分的理论依据,但工业现场往往缺乏必要的分析仪表和测量仪器,导致模型在工作点变动、给煤煤质变化等情况下出现较大偏差。数据驱动的建模方法可以直接利用锅炉运行数据建立模型,但相比机理建模缺乏理论基础,模型的可解释性不足。

目前的CFB锅炉机理建模采用的模型主要有1维模型、1.5维模型、2维模型和3维模型\cite{gungor2008}。1维模型最为简单,仅利用基本的物质和能量守恒定律,不考虑循环流化床锅炉径向差别。1.5维模型将径向截面分为稀相核和密相环,且核区与环区之间颗粒传递速率仅与颗粒浓度相关\cite{程从礼2002气固循环流化床能量最小多尺度环核}。2维模型将循环流化床锅炉炉膛底部细分为气固二相流,从而计算颗粒、气体的径向和轴向分布。3维模型基于Navier-Stokes方程,且考虑燃烧过程中的化学动力学,一般采用CFD(Computational Fluid Dynamics,计算流体力学)技术求解\cite{basu1999combustion}。1维、1.5维机理模型大多经过了现场试验验证,可以用于锅炉的控制和燃烧优化\cite{王新2008循环流化床动态建模及其预测控制研究}。2维模型和3维模型要求的参数相对较多,计算量很大,可以用于循环流化床锅炉的设计、动态培训仿真机开发和燃烧机理分析\cite{张晓磊2007循环流化床锅炉燃烧系统仿真}。清华大学在CFB锅炉的运行机理研究方面有很多成果,不仅构建了CFB锅炉的物料平衡理论、燃烧理论和传热理论,还提出了“定态设计”概念,并帮助国内多家锅炉厂家设计、开发CFB锅炉产品\cite{杨石2010基于流态重构的低能耗循环流化床锅炉技术,杨建华2009循环流化床锅炉改变床存量的燃烧试验}。

目前数据驱动的循环流化床锅炉建模方法有很多。除了传递函数和状态转移矩阵模型外,还有基于神经网络、支持向量机、模糊理论等模型\cite{赵伟杰2007循环流化床锅炉床温的控制特性,樊诚2007循环流化床锅炉燃烧过程建模研究,李易平1991循环流化床锅炉床温及床压建模}。神经网络模型可以较好地拟合锅炉的非线性特性,对神经网络模型的研究集中在BP(Back Propagation,反向传播)网络、RBF(Radial Basis Function,径向基函数)网络、小波网络等\cite{pulluri2005wavelet}。基于SVM(Support Vector Machine,支持向量机)的研究集中在SVM核函数和惩罚因子的选取\cite{RePEc:eee:energy:v:124:y:2017:i:c:p:284-294}。多数研究采用经验设置或者基于优化算法,通过两层结构选择一定意义上的最优参数组合。模糊模型,特别是T-S模型在建立锅炉燃烧系统模型方面也已有相当多的成果。此外,还有学者将神经网络和模糊理论的优点结合起来,建立锅炉的神经网络-模糊模型\cite{董湛波2012基于神经}。


控制策略设计的重要部分是设计控制器,包括基于PID的控制器、自适应控制、模糊控制、模型预测控制等。其中,由于PID控制器本身的局限性,基于PID的控制器的一个主要研究目标是提高其参数整定的鲁棒性和抗干扰能力。这方面的研究有鲁棒PID控制器优选法、基于DDE(Desired Dynamic Equation,预期动态方程)二自由度PID完全分散控制器、基于函数补偿和基于函数切换补偿的PID控制器等等\cite{徐峰2002鲁棒,郝玉春2012循环流化床燃烧系统的,郝玉春2013基于}。牛培峰等通过深入研究全系数自适应控制理论,在75$\,$\si[per-mode=symbol]{\tonne\per\hour}~和220$\,$\si[per-mode=symbol]{\tonne\per\hour}~循环流化床单元机组气压控制系统中,采用“空气-床温”反系统,实现了气压与床温的解耦;在135$\,$\si[per-mode=symbol]{\tonne\per\hour}~循环流化床汽温控制系统中,提出自适应串级控制方法,取得了较好的控制效果\cite{牛培峰2004大型国产循环流化床锅炉的汽温自适应串级控制系统,牛培峰2007循环流化床锅炉的汽压自适应控制系统}。数据挖掘技术也被应用到了流化床的模糊规则提取中,有学者利用聚类技术和模糊神经网络完成规则获取,同时针对循环流化床锅炉不确定性强的特点,将FBNC-PNN模糊神经自组织控制策略应用到锅炉主汽压力控制上,提高了控制性能\cite{王志强2003数据挖掘技术在循环流化床锅炉模糊控制系统中的应用}。王东风先后采用了基于多模型的广义预测控制和基于多模型的自适应预测控制对汽温进行控制,仿真实验得到良好的控制效果,并有很强的鲁棒性和适应性\cite{王东风2003循环流化床锅炉汽温系统的多模型自适应预测控制}。Park Juchirl等设计结合了神经网络和卡尔曼滤波,设计控制器来控制床高和物料循环率,实验研究结果显示这种控制器可以达到满意的控制效果\cite{Park2005The}。目前,控制器的设计越来越综合,出现了很多将不同控制器的设计思想综合起来的研究工作。

除了选用先进控制器外,研究锅炉的燃烧机理也有助于设计更好的控制策略。基于即燃碳机理模型,建立基于即燃碳平衡的给煤控制、风煤优化配比等策略,可以大大提高了机组的负荷调节速度,同时维持主汽压力的稳定。基于活性石灰石模型,构造~\ce{SO2}质量浓度预测模型,通过优化钙硫比配比,可以提高~\ce{SO2}质量浓度的稳定性\cite{高明明2014600MW}。杨石等提出基于流态重构的循环流化床节能燃烧技术,可以让床压运行在最低优化床压降,这样可提高锅炉燃烧效率,降低炉内物料浓度和受热面磨损,并减轻一次风机耗电量,显著降低了CFB锅炉的发电成本\cite{杨石2010基于流态重构的低能耗循环流化床锅炉技术}。

循环流化床锅炉的燃烧优化方面也有很多成果。1994年,Kortela提出分层控制的思想,上层为稳态运行优化层,下层为燃烧控制层\cite{kortela1994modelling},优化建立在稳定控制的基础之上。目前,各过程控制公司的过程优化商用软件一般也采用类似的两层结构。陈亮在神经元控制器的基础上,提出了多目标优化的单神经元PID控制算法,利用有监督的学习策略对神经元的权值进行调整\cite{陈亮2005循环流化床锅炉建模及其智能控制系统的研究}。童一飞构建了2层基于广义预测控制的燃烧优化系统,底层实现床温和主蒸汽压力的动态控制,顶层实现经济性能指标的稳态优化,从而实时降低煤耗、调高经济效益\cite{童一飞2010基于广义预测控制的循环流化床锅炉燃烧过程多目标优化控制策略}。在设计经济性能指标时,除了计算电耗和煤耗外,还考虑可加入国家相关政策对相关污染物排放量的奖惩项\cite{谢磊2016循环流化床锅炉燃烧过程预测控制与经济性能优化}。

\subsection{\ce{NOx}软测量研究发展现状}
\ce{NOx}包括~\ce{N2O}、\ce{NO}、\ce{N2O2}、\ce{N2O3}、\ce{NO2}、\ce{N2O4}、\ce{N2O5}等多种物质。也有学者将~\ce{NOx}的范围仅限于~\ce{NO}和~\ce{NO2},因为所有种类的氮氧化物中~\ce{NO}和~\ce{NO2}占的比重最大,且其主要来源为人类的生产生活\cite{Agency1999Nitrogen}。与建立CFB锅炉燃烧系统模型类似,\ce{NOx}软测量建模的研究也分为基于机理的建模方法和数据驱动的建模方法,此外还有结合两种建模方法的半机理模型。事实上,在利用机理法建立CFB锅炉燃烧系统模型的过程中,往往就包含了建立~\ce{NOx}生成的子模型的过程。

CFB锅炉中的~\ce{NOx}生成机理非常复杂,且不同的氮氧化物可以在一定条件下互相转化,建立完全描述其生成过程和分布的机理模型十分困难。基于燃烧过程生成的~\ce{NOx}大多数为~\ce{NO},少部分为~\ce{NO2},其它种类的氮氧化物非常少这一特点,不少研究工作都集中在建立~\ce{NO}生成模型上。这类研究又分为两种:一种根据锅炉内的总体反应过程,只关心煤、石灰石和炉内的反应催化剂,不关注反应的中间过程;另外一种方法则详细描述炉内实际发生的化学反应\cite{johnsson1990modelling}。对CFB锅炉来说,共计有28种物质、106个化学反应被用来建立~\ce{NOx}生成的模型\cite{talukdar1995simplified}。另一种氮氧化物~\ce{N2O},虽然在燃烧生成的氮氧化物中含量很少,但由于其温室效应作用是~\ce{CO2}的200多倍,且在自然界中的半衰期长达100-150年\footnote{化学物质的半衰期指其浓度经某种反应降到初始值的一半时经过的时间},也有研究者分别建立~\ce{NO}和~\ce{N2O}生成量的模型\cite{liukkonen2012dynamic}。

随着国家对~\ce{NOx}排放提出更高要求,不少燃煤锅炉包括循环流化床锅炉开始附加脱硝设备对~\ce{NOX}进行处理,其中应用最广的是基于化学反应的脱硝设备,其处理方法主要有SCR(Selective Catalytic Reduction,选择性催化还原)和SNCR(Selective Non-Catalytic Reduction,选择性非催化还原)。由于SCR的效率与催化剂有关,对~\ce{NOx}排放量机理建模需要考虑不同催化剂的活性和催化剂对~\ce{NOx}还原机理的作用。即使对于不需要催化剂的SNCR脱硝设备,也需要考虑还原剂喷射位置、方式以及氨逃逸等对还原反应的影响\cite{李穹2013sncr}。以上种种因素都加大了根据化学反应机理建立~\ce{NOx}排放量模型的难度。

数据驱动的建模方法不考虑CFB锅炉和脱硝设备内的种种化学反应,直接利用锅炉的运行数据建立~\ce{NOx}排放量模型。这些模型主要有基于统计回归方法的回归模型、神经网络模型和SVM模型。这三种模型各有优缺点:回归模型计算速度最快,模型解释性最好,但对数据质量有一定要求,且不能描述系统中的非线性特性;神经网络模型计算速度较慢,可以描述非线性特征,容易出现过拟合,模型参数不易选择,解释性不好;SVM模型也能描述模型的非线性,相对于神经网络更适合小样本建模,但在样本数据量较大时计算复杂度迅速上升。

浙江大学的白卫东等利用图像处理技术获得锅炉内的温度场,并将温度场加入回归模型的输入变量矩阵建立~\ce{NOx}排放量模型,但实际现场中锅炉内部火焰图像很难采集,且图像处理工作会消耗大量的计算资源\cite{白卫东2004统计回归方法在电站锅炉氮氧化物排放量监测中的应用}。孔亮等将Kriging插值算法应用到~\ce{NOx}建模,该模型的内插和外推能力优于神经网络模型\cite{孔亮2008基于}。

由于BP神经网络具有算法简单的优点,也有很多学者将BP网络模型用于~\ce{NOx}排放量的建模。针对BP网络本身的缺点,结合Adaboost算法的BP网络可以提高模型的预测性能。考虑到过程中时延的影响,有学者分别建立了三种不同的神经网络模型,其中考虑输入变量延时的神经网络预测性能最好\cite{Li2003Neural}。此外,随着机器学习、深度学习等技术的研究发展,也有学者将这些方法应用到~\ce{NOx}建模中,取得了较好的效果\cite{li2016prediction}。

王雅斌等以高斯径向基函数作为核函数,建立电站锅炉~\ce{NOx}排放量的SVM模型,模型精确度较高,但模型训练和测试所用数据都来自锅炉稳态情况下的正交实验,且数据量较小(训练样本90组,测试样本15组)。常用的优化方法如GA(Genetic Algorithm,遗传算法)、网格搜索、PSO(Particle Swarm Optimization,粒子群优化算法)、ACO(Ant Colony Optimization,蚁群优化算法)等都可以用来解决SVM参数选择问题。其中网格搜索速度最快,但其模型质量低于其它几种优化算法\cite{王雅彬2012基于支持向量机的电站锅炉}。

\subsection{本课题组研究进展}

课题组的研究方向一直围绕着工业过程的先进控制与优化,在工业锅炉控制方案和软测量方面都有一定的积累。

张旭东提出一种基于工业锅炉运行数据提取对象脉冲响应序列的方法,并利用该序列建立锅炉一次风压的MAC(Model Algorithm Control,模型算法控制)控制器,仿真结果显示这种设计方法在高噪声系统中优于PID控制器\cite{薛美盛2012电站锅炉一次风压系统模型算法控制仿真研究}。考虑到锅炉中送风机、引风机均有甲乙两侧,胡祖辉提出双侧变增益广义预测控制,并设计了三种双侧控制量协调策略,投运后锅炉氧量波动范围接近~$\pm$0.5$\,$\si{\percent}\cite{胡祖辉2015母管制锅炉先进控制策略研究及应用}。

王振将GPC(Generalized Predictive Control,广义预测控制算法)应用到锅炉主汽温度控制上,设计了主汽流量前馈补偿、GPC~-~PID串级控制策略,串级回路的内回路控制减温水流量,实际投运后主汽温度的波动范围降低到~$\pm$6$\,$\si{\degreeCelsius}。针对GPC控制器参数整定的问题,王振又提出了$\gamma$~-~SGPC算法,在分析算法的缺陷后,进一步提出了$\gamma$~-~${\gamma}'$~-~SGPC,实现了$\gamma$的自整定\cite{王振2012基于负荷前馈补偿的主汽温串级广义预测控制,樊培利2011改进的γ一}。针对吹灰操作对锅炉主汽温度的影响,胡棋设计了吹灰过程专家监督系统,在监测到吹灰开始后,利用专家知识库推理计算所需的补偿减温水量,该系统投运后主汽温度波动范围降到~$\pm$3$\,$\si{\degreeCelsius}\cite{胡棋2014基于吹灰专家监督系统的主汽温度广义预测控制}。

薛美盛等提出一种工业锅炉燃烧过程优化方案,考虑到锅炉负荷和炉膛温度能够表征当时锅炉的燃烧状态,基于锅炉负荷设定烟道氧量,基于炉膛温度设定风煤比,在线优化器投运后锅炉效率提高了8$\,$\si{\percent}\cite{薛美盛2001工业锅炉在线燃烧优化}。刘长远等开发了一套具有较高可移植性的电站锅炉先进控制系统,先进控制系统通过OPC(OLE for Process Control,用于过程控制的对象链接与嵌入)协议与现场DCS进行通讯,投运后明显改善了风烟系统和汽水系统的控制效果\cite{刘长远2012电站锅炉先进控制系统的开发与应用}。

崔宇结合局部学习和LSSVM(Least Squares Support Vector Machine,最小二乘支持向量机),建立了锅炉飞灰含碳量软测量模型,并提出鲁棒PCA的加权LSSVM设计方案,最后利用该软测量模型对锅炉进行多目标稳态优化\cite{崔宇2009局部}。李先知在开发污水处理过程远程监控系统的基础上,利用LSSVM建立了COD (Chemical Oxygen Demand,出水化学需氧量)、TN (Total Nitrogen,出水总氮含量)的软测量模型,在模型精度无法满足时,采用网格算法对模型参数寻优\cite{李先知2016污水处理过程远程监控系统设计与支持向量机技术应用研究}。

\section{本文内容安排}

本文在分析CFB锅炉燃烧系统控制目标和控制策略的基础上,设计实现了基于广义预测控制器的先进控制系统,并将其应用到某电站锅炉。针对现场~\ce{NOx}~传感器经常出现故障影响自动控制投运效果这一问题,利用多元回归分析建立~\ce{NOx}排放量软测量模型,分析不同自变量选择方法对软测量模型的影响。


本文结构安排如下:

第一章\emph{绪论}。介绍本文研究背景及研究意义,总结CFB锅炉燃烧系统控制相关研究和~\ce{NOx}~软测量相关研究现状,描述课题组在锅炉控制与优化和软测量两个领域的相关研究成果。最后是全文研究内容和结构安排。

第二章\emph{现场情况简介}。首先是CFB锅炉,这是本文的控制对象,简单描述其设计参数、整体布置和各个分系统的工艺流程。然后是CS3000,这是电站锅炉采用的DCS系统。接下来分析燃烧系统现有控制水平,包含主汽压力、床温、床层压差三个回路。最后结合现场实测数据分析~\ce{NOx}~测量仪表存在的问题以及对控制系统的影响。

第三章\emph{先进控制系统设计与实现}。介绍广义预测控制算法,针对第二章中控制策略的不足,设计燃烧系统先进控制策略。接下来是先进控制系统的软硬件设计和实现。最后介绍将先进控制系统纳入DCS所需的组态修改工作。

第四章\emph{循环流化床锅炉燃烧系统先进控制应用}。首先介绍数据预处理和模型辨识方法。由过程运行数据建立模型后,介绍控制器各参数的对控制效果的影响及回路整定的顺序,最后给出燃烧系统各回路的控制效果。

第五章\emph{氮氧化物软测量}。在选定模型为线性回归模型后,基于~\ce{NOx}~生成和SNCR脱硝机理,建立模型输入量集合。随后介绍回归变量的选择方法,并比较不同选择方法在不同建模数据中的性能。设计了结合滚动窗思想的软测量模型,接着比较了滚动窗参数对软测量模型性能的影响。最后比较软测量模型的预测结果与实际测量值。

第六章\emph{总结与展望}。总结论文的主要工作和创新点,指出研究中的不足之处,对后续研究工作进行展望。

