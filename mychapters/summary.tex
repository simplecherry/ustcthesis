\chapter{总结与展望}

循环流化床锅炉燃烧系统的控制水平直接影响效率。本文以某电厂锅炉的燃烧系统为研究对象,构建了一套基于GPC的先进控制系统。现场运行数据表明投运先进控制系统之后,锅炉的燃烧效率和一次风机电耗明显降低。针对现场~\ce{NOx}~测量仪表频繁故障的问题,设计了一种基于多元线性回归的~\ce{NOx}~软测量模型,仿真结果显示该模型能够在较长的时间内准确地估计锅炉~\ce{NOx}~排放量。

\section{研究工作总结}


\emph{1.设计并实现了先进控制系统,完成了DCS系统组态修改,保障先进控制的安全运行。}

为了尽可能减少先进控制系统对原有DCS控制的影响,设计先进控制系统通过OPC接口与DCS连接。先进控制软件分不同功能模块设计,并利用配置文件保证了系统的可移植性。在先进控制系统与DCS连接后,添加组态文件,使先进控制投切时满足现场工艺要求,且与原有控制方式无扰切换。最后增加了先进控制在DCS端的监控画面,使先控也能像DCS原有控制方式一样便于操作人员监控和操作,见第~\ref{chap:design}~章。

\emph{2.设计了基于GPC的燃烧系统控制策略,并将其应用到CFB锅炉,取得了较好的控制效果。}

针对现场原有控制策略的不足,基于阶梯式广义预测控制,重新设计了燃烧系统的控制策略。对现场各回路建立数学模型后,进行控制器参数整定,将新的控制策略应用到现场中。经过一定时间的投运后,燃烧系统各回路的被控量波动幅度明显降低,满足了现场控制需求。经估算,投运先控能够节省相当一部分的燃煤和辅机用电,见第~\ref{chap:application}~章。

\emph{3.基于多元线性回归原理,结合滚动窗方法,建立了一种动态更新的~\ce{NOx}排放量软测量模型,并分析了不同自变量选择方法对模型性能的影响。}

在利用多元线性回归方法建立静态~\ce{NOx}软测量模型的基础上,比较了不同自变量选择策略选择的自变量以及模型性能。为了让软测量模型适应系统工况变化,结合滚动窗方法建立随系统运行不断更新的~\ce{NOx}排放量模型,并研究了窗口长度和滚动距离对模型的影响。最后,结合现场数据,对比软测量模型与现场实际输出值,软测量模型能够较好的估计现场~\ce{NOx}排放量,均方根误差约为1.79$\,$\si{mg/m^3},见第~\ref{chap:softsensor}~章。


\section{后续研究工作展望}
\emph{1.在建立CFB锅炉数学模型时采用的是线性模型,考虑到锅炉的非线性特性,建立非线性模型或多模型可以更好地预测锅炉的动态过程。}

CFB锅炉的燃烧过程非常复杂,本文将控制对象在系统工作点附近简化为带纯滞后的一阶惯性过程或积分过程,实际上不能保证模型在全部工作区间内的准确性。如果能够建立对象的非线性模型或多模型,可以在不同的工作点都能保证模型的精度。

\emph{2.现场燃烧系统实际上采用的仍是单回路控制,由于燃烧系统各回路之间耦合较强,可以针对其建立解耦控制策略。}

燃烧系统的一次风量和给煤量对锅炉的床温、床层压差和主汽压力都有很大影响。目前采用的控制策略虽然取得了不错的控制效果,但床层压差回路的控制精度还可以进一步提高。因此,可以设计解耦控制策略,降低不同回路之间的干扰,提高系统的控制水平。

\emph{3.锅炉负荷对主汽压力有较大影响,建立锅炉负荷预测模型可以提高主汽压力的控制水平。}

本文的控制策略将锅炉负荷作为主汽压力回路的干扰量,并对其做静态前馈补偿。实际上,主汽流量对主汽压力的作用要快于给煤量对主汽压力的作用,给煤量需要提前变化才能保证对干扰的补偿。由于厂区的生产有一定的计划和规律,可以根据历史数据建立锅炉负荷预测模型,降低外部干扰对控制系统的影响。

\emph{4.利用更多的数据、新的信息准则和辨识算法建立~\ce{NOx}软测量模型。}

本文受现场测点限制,在建立模型候选输入集的时候只采用了10个变量。实际上~\ce{NOx}排放量与很多因素相关,在测量点较多的时候可以建立更大的模型输入集。本文在计算模型参数采用的是基本的最小二乘方法,建立的模型是线性模型。实际上现场数据不一定能够保证最小二乘法是最优无偏估计,针对现场噪声特性选用其它辨识方法也许能够得到更好的结果。另外,非线性模型更接近实际~\ce{NOx}排放特性,下一步也可以建立非线性~\ce{NOx}软测量模型。

