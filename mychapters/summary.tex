\chapter{总结与展望}

循环流化床锅炉燃烧系统的控制水平直接影响锅炉的热效率。本文以某电厂锅炉的燃烧系统为研究对象,构建了一套基于广义预测控制的先进控制系统,并成功投运于现场。现场的运行数据表明:投运先进控制系统之后,锅炉的单位蒸汽煤耗和一次风机电耗明显降低。针对现场~\ce{NOx}~测量仪表频繁故障的问题,设计了一种基于多元线性回归的~\ce{NOx}~软测量模型,仿真结果显示该模型能够在较长的时间内准确地估计锅炉~\ce{NOx}~排放量。

\section{研究工作总结}
\emph{1.设计并实现了CFB锅炉燃烧先进控制系统,完成了DCS系统组态修改。}

为了尽可能减少燃烧先进控制系统对原有DCS控制的影响,设计先进控制系统通过OPC接口与DCS连接。先进控制软件按照不同功能分为数据通讯、数据趋势、控制算法三个模块设计,并利用回路配置文件和数据点配置文件保证了软件的可移植性。采用额外采购的一台工控机作为先进控制系统运行的硬件平台,且配置随时更新。在先进控制系统与DCS连接后,添加组态文件,使先进控制投切时满足工艺条件要求,且与原有控制方式无扰切换,增加了先进控制在DCS端的监控画面,见第~\ref{chap:design}~章。

\emph{2.设计了基于GPC的燃烧系统控制策略,并成功应用到电厂CFB锅炉,取得了较好的控制效果。}

针对现场原有控制策略的不足,基于阶梯式广义预测控制,重新设计了燃烧系统的控制策略。利用相关系数法和批处理最小二乘建立各回路的数学模型,进行控制器参数整定,将新的控制策略应用到现场中。经过一定时间的投运后,燃烧系统各回路的被控量波动幅度明显降低,满足了现场控制需求。经估算,投运先控能够节省相当一部分的燃煤和辅机用电,见第~\ref{chap:application}~章。

\emph{3.基于多元线性回归原理,结合滚动窗方法,建立了一种动态更新的~\ce{NOx}排放量软测量模型,并分析了不同自变量选择方法对模型性能的影响。}

在利用多元线性回归方法建立静态~\ce{NOx}软测量模型的基础上,比较了不同自变量选择策略选择的自变量以及模型性能。为了让软测量模型适应系统工况变化,结合滚动窗方法建立随系统运行不断更新的~\ce{NOx}排放量模型,并研究了窗口长度和滚动距离对模型的影响。最后,结合现场数据,对比软测量模型与现场实际输出值,软测量模型能够较好的估计现场~\ce{NOx}排放量,均方根误差约为1.79$\,$\si{mg/m^3},见第~\ref{chap:softsensor}~章。


\section{后续研究工作展望}
\emph{1.在建立CFB锅炉数学模型时采用的是线性模型,考虑到锅炉的非线性特性,建立非线性模型或多模型可以更好地预测锅炉的动态过程。}

CFB锅炉的燃烧过程非常复杂,本文将控制对象在系统工作点附近简化为带纯滞后的一阶惯性过程或积分过程,结合鲁棒性较强的GPC控制器,取得了较好的控制效果。如果能够建立对象的非线性模型或多模型,可以更准确地预测未来系统输出,进一步提高控制水平。

\emph{2.现场燃烧系统采用的仍是单回路控制,由于燃烧系统各回路之间耦合较强,可以针对其建立解耦控制策略。}

燃烧系统的一次风量和给煤量对锅炉的床温、床层压差和主汽压力都有较大影响。目前采用的控制策略虽然取得了不错的控制效果,但床层压差回路的控制精度还可以进一步提高。因此,设计合适的解耦控制策略,降低不同回路之间的干扰,提高系统的控制水平。

\emph{3.锅炉负荷对主汽压力有较大影响,建立锅炉负荷预测模型可以提高主汽压力的控制水平。}

本文的控制策略将锅炉负荷作为主汽压力回路的干扰量,并对其做静态前馈补偿,在锅炉负荷波动较大时依然保证主汽压力达到国标要求。如果锅炉负荷对主汽压力的作用快于给煤量对主汽压力的作用,给煤量实际上需要提前变化才能保证对干扰的补偿。由于厂区的生产有一定的计划和规律,可以根据历史数据建立锅炉负荷预测模型,进一步降低主汽压力的偏差。

\emph{4.利用更多的数据、新的信息准则建立~\ce{NOx}软测量模型。}

本文受现场测点限制,在建立模型候选输入集的时候只采用了10个变量。实际上~\ce{NOx}排放量与很多因素相关,在测量点较多的时候可以建立更大的模型输入集,也许能找到更多与~\ce{NOx}排放相关的模型。本文采用最小二乘方法建立线性模型,结合动态的自变量选择机制,能做出较为准确的~\ce{NOx}排放量预测。在对软测量模型实时性能要求不高的情况下,可以采用更复杂的信息准则来获得模型输入和参数。另外,非线性模型更接近实际~\ce{NOx}排放特性,下一步也可以建立非线性~\ce{NOx}软测量模型。

